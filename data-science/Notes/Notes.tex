\documentclass[12pt]{article}

% Packages
\usepackage[utf8]{inputenc} % UTF-8 encoding
\usepackage{amsmath, amssymb} % Math symbols
\usepackage{graphicx} % For including graphics
\usepackage{hyperref} % For hyperlinks
\usepackage{xcolor} % For colored text
\usepackage{geometry} % For page layout
\geometry{a4paper, margin=1in}

% Customizations
\setlength{\parindent}{0pt} % No paragraph indentation
\setlength{\parskip}{1em} % Add space between paragraphs
\renewcommand{\baselinestretch}{1.2} % Line spacing

% Title and Author
\title{Data Science Notes}
\author{Bhavjot Khurana}
\date{\today}

\begin{document}

\maketitle
\tableofcontents
\newpage

\section{Introduction}
Using the freecodecamp.ord video as a reference (link: https://www.youtube.com/watch?v=XU5pw3QRYjQ). This document contains notes on data science topics covered in the video.

\section{Topic 1: Linear Regression}
\subsection{Definition}
Linear regression models the expected value of a response variable \(y\) as a linear function of a single predictor \(x\):
\[
y_i = \beta_0 + \beta_1 x_i + \varepsilon_i,\quad \varepsilon_i \sim \text{Normal}(0, \sigma^2)
\]
It assumes the errors \(\varepsilon_i\) are independent, normally distributed, and have constant variance.

\subsection{Key Formulas}
Ordinary least squares (OLS) estimates the slope and intercept by minimizing the sum of squared residuals:
\[
\hat{y}_i = \hat{\beta}_0 + \hat{\beta}_1 x_i
\]
\[
\hat{\beta}_1 = \frac{\sum_{i=1}^{n} (x_i - \bar{x})(y_i - \bar{y})}{\sum_{i=1}^{n} (x_i - \bar{x})^2}, \qquad
\hat{\beta}_0 = \bar{y} - \hat{\beta}_1 \bar{x}
\]

\subsection{Error Function}
The residual for observation \(i\) captures the prediction error:
\[
e_i = y_i - \hat{y}_i
\]
Residuals tell us how far each point lies from the regression line; examining their pattern helps spot outliers or violations of model assumptions.

\subsection{Mean Squared Error (MSE)}
Mean Squared Error is the average of squared residuals:
\[
\text{MSE} = \frac{1}{n} \sum_{i=1}^{n} (y_i - \hat{y}_i)^2
\]
Errors are squared to penalize large deviations more heavily, eliminate sign cancellation, and produce a smooth, differentiable loss function that calculus-based solvers can optimize.

\subsection{How the Line is Fitted}
Fitting the line involves solving the optimization problem:
\[
\min_{\beta_0, \beta_1} \sum_{i=1}^{n} (y_i - \beta_0 - \beta_1 x_i)^2
\]
Setting the partial derivatives to zero yields the normal equations (matrix form: \((X^\top X)\hat{\beta} = X^\top y\)). Software uses closed-form solutions for small problems or matrix decompositions (QR or SVD) for numerical stability. Gradient methods (e.g., gradient descent) offer scalable alternatives for very large datasets.

\subsection{Hypothesis Testing and p-values}
To test whether \(x\) helps explain \(y\), use the t-test for the slope. The null hypothesis \(H_0:\beta_1 = 0\) indicates no linear relationship. The statistic
\[
t = \frac{\hat{\beta}_1}{\text{SE}(\hat{\beta}_1)}
\]
follows a t-distribution with \(n-2\) degrees of freedom. A small p-value suggests the predictor provides statistically significant explanatory power.

\subsection{Residual Standard Error (RSS and TSS)}
Two sums of squares quantify variability:
\[
\text{RSS} = \sum_{i=1}^{n} (y_i - \hat{y}_i)^2, \qquad
\text{TSS} = \sum_{i=1}^{n} (y_i - \bar{y})^2
\]
Residual Standard Error (RSE) estimates the typical size of residuals in the units of \(y\):
\[
\text{RSE} = \sqrt{\frac{\text{RSS}}{n-2}}
\]
Lower RSE indicates tighter fit around the regression line.

\subsection{Interpretations}
\begin{itemize}
    \item \textbf{Slope \(\hat{\beta}_1\)}: Expected change in \(y\) for a one-unit increase in \(x\).
    \item \textbf{Intercept \(\hat{\beta}_0\)}: Expected value of \(y\) when \(x = 0\), useful when \(x=0\) is meaningful.
    \item \textbf{\(R^2 = 1 - \frac{\text{RSS}}{\text{TSS}}\)}: Proportion of variance in \(y\) explained by the model.
    \item \textbf{Practical use}: Combine coefficient estimates with confidence or prediction intervals to communicate both central tendency and uncertainty in predictions.
\end{itemize}

\subsection{Example: Simple Linear Regression Output}
Using the \texttt{Advertising} dataset (TV spend predicting sales), the \texttt{statsmodels} summary reports the following key statistics:

\begin{table}[h]
    \centering
    \begin{tabular}{lcc}
        \hline
        Statistic & Estimate & Interpretation \\ 
        \hline
        Intercept (\(\hat{\beta}_0\)) & 7.03 & Baseline sales when TV spend is \$0k. \\ 
        TV coefficient (\(\hat{\beta}_1\)) & 0.0475 & Each \$1k on TV adds approximately 0.048k units sold. \\ 
        $R^2$ & 0.612 & 61.2\% of variance in sales explained by TV. \\ 
        F-statistic & 312.1 (p $< 10^{-40}$) & The model is highly significant overall. \\ 
        \hline
    \end{tabular}
\end{table}

The printed OLS table also shows small standard errors and very large t-statistics for the slope, so the p-value for the TV coefficient is essentially zero. Residual diagnostics (Omnibus, Jarque-Bera, Durbin-Watson) indicate roughly normal residuals and limited autocorrelation for this example.

\section{Topic 2: Multiple Linear Regression}
\subsection{Definition}
Multiple Linear Regression (MLR) extends the linear model to \(p\) predictors:
\[
y_i = \beta_0 + \beta_1 x_{i1} + \dots + \beta_p x_{ip} + \varepsilon_i
\]
Coefficients measure the expected change in \(y\) for a one-unit change in the corresponding predictor while holding others constant.

\subsection{F-statistic and Interpretation}
The model-wide F-test compares the explained variance to unexplained variance:
\[
F = \frac{(\text{TSS} - \text{RSS}) / p}{\text{RSS} / (n - p - 1)}
\]
A large F-statistic (with a small p-value) indicates that, taken together, the predictors explain significantly more variation than an intercept-only model. If the F-test is not significant, it implies the collective set of predictors may not offer meaningful predictive power beyond the mean of \(y\).

\subsection{Example: Multiple Linear Regression Output}
Fitting TV, radio, and newspaper spend simultaneously to predict sales yields the following summary highlights:

\begin{table}[h]
    \centering
    \begin{tabular}{lcc}
        \hline
        Statistic & Estimate & Interpretation \\ 
        \hline
        Intercept & 2.94 & Baseline sales when all spends are zero. \\ 
        TV coefficient & 0.0458 & Positive, highly significant effect (p $< 10^{-50}$). \\ 
        Radio coefficient & 0.1885 & Strong positive effect (p $< 10^{-40}$). \\ 
        Newspaper coefficient & -0.0010 & Not significant (\(p \approx 0.86\)). \\ 
        $R^2$ & 0.897 & 89.7\% of sales variance explained jointly. \\ 
        F-statistic & 570.3 (p $< 10^{-90}$) & Model is overwhelmingly significant. \\ 
        \hline
    \end{tabular}
\end{table}

Compared with the simple-regression model, $R^2$ jumps from 0.61 to 0.90, showing the value of additional predictors. However, the newspaper coefficient's large p-value signals it may not contribute meaningfully, suggesting a model refinement (e.g., dropping the variable) could be appropriate.

\section{Topic 3: Machine Learning}
\subsection{Supervised Learning}
Explain supervised learning here.

\subsection{Unsupervised Learning}
Explain unsupervised learning here.

\section{Conclusion}
Summarize your notes here.

\end{document}
